\csection{Заключение}

В данном разделе будут подведены итоги
по проделанной в рамках данного курсового проекта
работы по написанию драйвера USB клавиатуры
для ядра Linux.

В ходе выполнения проекты были получены множественные
теоретические сведения, связанные с общими положениями
системного программирования, 
некоторых аппаратных решений в современных вычислительных
системах.

Помимо общих сведений, например,
о системе прямого доступа к памяти или
интерфейсе универсальной последовательной шины,
были приобретены специфические теоретические знания,
относящиеся к экосистеме ядра Linux, 
а также к непосредственной разработке 
модулей ядра для обслуживания периферийных устройств:
принципы работы модулей ядра в системе, устройство подсистем ввода и USB,
а также использование систем сборки Make и Kbuild.

Получены практические навыки в написании модулей ядра Linux,
необходимых для работы подсоединяемых устройств.
Также к приобретенным навыкам можно отнести
событийное программирование для некоторых подсистем ядра,
а также способы синхронизации памяти и устранение гонок данных
при разработке многопоточного и асинхронного программного обеспечения.

Опыт разработки модулей ядра Linux во многом отличается
от системного программирования, изучаемого в рамках теоретического
курса предмета, и открывает множество возможностей по дальнейшему изучению
тем, связанных с операционными системами и системным программированием,
например, разработка системного программного обеспечения для серверов,
или программирование микроконтроллеров и систем реального времени.

Опыт в разработке драйверов устройств, связанный с изучением
технической документации, стандартов и соглашений,
полезен для дальнейшей самостоятельной работы
в узконаправленных предметных областях,
возмножности по изучению которых ограничиваются документацией и 
ресурсами соответствующего сообщества.
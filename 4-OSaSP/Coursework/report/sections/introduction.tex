\csection{Введение}

В настоящее время наблюдается стремительный рост в области периферийных устройств:
клавиатуры, компьютерные мыши, джойстики, веб-камеры, динамики,
наушники, микрофоны, флеш-накопители, принтеры, сканеры,
внешние диски и видеокарты -- лишь немногая часть из того, что на сегодняшний
день используется человеком в повседневной жизни.
Каждое из этих устройтв имеет собственную реализацию,
определяемую как назначением, так и производителем;
порой даже те устройства, что имеют общее назначение,
могут различаться по своей технической реализации.
Однако большинство данных устройств имеют одну общую деталь,
что позволяет использовать их повсеместно,
и этой деталью является интерфейс последовательной шины USB.

С появлением последовательного интерфейса USB, а также его
внедрения в вычислительную технику,
была решена проблема существования множества интерфейсов для
периферии, что существенно облегчило жизнь как производителям
устройств и вычислительной техники, так и обычным пользователям.

Данное изменение не просто породило множество устройств,
использующих общую технологию, но и потребовало написание
не меньшего количества системного кода, поддерживающего их работу.

Одним из видов такого системного программного обеспечения является 
драйвер -- посредник между ядром операционной
системы и аппаратной реализацией устройства, 
предоставляющий пользователю доступ к пользованию устройством.

Тема данного курсового проекта -- разработка драйвера
устройства USB клавиатуры. Актуальность данного проекта обуславливается
не только большой востребованностью программного обеспечения,
поддерживающего работу USB-устройств, 
но и огромной теоретической и практической ценностью самой задачи,
предоставляющей возможность изучить область разработки,
не рассматриваемую в рамках основного курса дисциплины по системному программированию.

Ввиду того, что разработка драйвера во многом зависит не столько от 
самого устройства устройства, столько от целевой платформы, для которой он пишется,
следует заранее обозначить, что данный драйвер будет разрабатываться
под операционные системы на базе ядра Linux.
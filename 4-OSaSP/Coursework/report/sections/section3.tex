\section{Функциональное проектирование}

Ниже будут представлены результаты проектирования
каждой функциональной группы, описанной в разделе \ref{system-design}.

\subsection{Группа вспомогательных функций} \label{utility-group-design}

Ниже представлено объявление функции инициализации памяти
для структуры данных \texttt{usbkeyboard}:
\begin{verbatim}
    int usbkeyboard_alloc_memory(struct usb_device* dev, 
                                 struct usbkeyboard* kbd);
\end{verbatim}

Данная функция инициализирует все динамические поля структуры,
связанные с подсистемой USB (память для URB и DMA).

Ниже представлено объявление функции деинициализации памяти для
структуры данных \texttt{usbkeyboard}:
\begin{verbatim}
    void usbkeyboard_free_memory(struct usb_device* dev, 
                                 struct usbkeyboard* kbd);
\end{verbatim}

Данная функция выполняет работу, противоположную работе функции
инициализации \texttt{usbkeyboard\_alloc\_memory}.

\subsection{Группа контрольных событий}

Ниже представлено объявление функции инициализации драйвера
при подключении устройства:
\begin{verbatim}
    int usbkeyboard_probe(struct usb_interface* intf, 
                          const struct usb_device_id* id);
\end{verbatim}

Данная функция выполняет все действия, необходимые для регистрации
устройства клавиатуры в системе и его последующего обслуживания драйвером.

Ниже представлено объявление функции деинициализации драйвера
при отключении устройства:
\begin{verbatim}
    void usbkeyboard_disconnect(struct usb_interface *intf);
\end{verbatim}

Данная функция выполняет работу, противоположную работе функции 
инициализации драйвера \texttt{usbkeyboard\_probe}.
%
\pagebreak

Ниже представлено объявление функции запуска обработки нажатий 
клавиатуры при открытии устройства ввода:
\begin{verbatim}
    int usbkeyboard_open(struct input_dev* dev);
\end{verbatim}

Данная функция регистрирует обработку нажатия клавиш
клавиатуры при открытии соответствующего ей устройства ввода.

Ниже представлено объявление функции остановки обработки нажатий
клавиатуры при закрытии устройства ввода:
\begin{verbatim}
    void usbkeyboard_close(struct input_dev* dev);
\end{verbatim}

Данная функция выполняет работу, противоположную работе функции
запуска обработки нажатий клавиатуры \texttt{usbkeyboard\_open}.

\subsection{Группа управления нажатиями клавиш}

Ниже представлено объявление функции обработки отчетов о нажатии 
клавиш клавиатуры:
\begin{verbatim}
    void usbkeyboard_irq(struct urb* urb);
\end{verbatim}

Данная функция сравнивает предыдущее состояние контекста \texttt{urb},
и, на основании этого сравнения, сообщает подсистеме ввода о новых нажатиях
и отпусканиях клавиш, если они произошли.

\subsection{Группа управления состоянием светодиодов}

Ниже представлено объявление функции управления состоянием
светодиодов на клавиатуре:
\begin{verbatim}
    void usbkeyboard_led(struct urb* urb);
\end{verbatim}

Данная функция отправляет устройству клавиатуры 
упрвавляющий запрос на изменение состояния светодиодов.

Ниже представлено объявление функции обработки события изменения
состояния светодиодов в подсистеме ввода:
\begin{verbatim}
    int usbkeyboard_event(struct input_dev* dev, 
                          unsigned int type, 
                          unsigned int code, int value);
\end{verbatim}

Данная функция преобразует сведения о событии в управляющий
запрос на изменение состояния светодиодов клавиатуры и 
регистрирует его отправление.
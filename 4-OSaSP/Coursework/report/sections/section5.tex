\section{Руководство пользователя}

% Как как модули ядра не имеют пользовательского интерфейса, 
% а реализованный модуль не содержит специальных параметров загрузки,
% руководство пользователя должно предоставить сведения о способе поставки
% драйвера, а также о способе его установки в системе.

% \subsection{Поставка программного обеспечения}

% Существует несколько основных видов поставки программного обеспечения:
% \begin{enumerate}
%       \item Поставка в закрытом виде. 
%             В этом случае поставляется бинарного выпуск продукта, 
%             также называемый \emph{блобом (blob, binary object)}.
%             Как правило, таким образом поставляется проприетарные решения,
%             разработчики которого из коммерческих соображений не желают открывать свои технологии.
%       \item Поставка в открытом виде.
%             В этом случае продукт поставляется исходный код продукта с необходимыми инструментами сборки,
%             либо к бинарному выпуску продукта прилагается открытый исходный код.
%             Как правило, таким способом поставляется программное обеспечение от сообщества,
%             либо программное обеспечение, разрабатываемое организациями на основе открытых технологий.
% \end{enumerate}

% Помимо вопроса о виде, в котором распространяется программное обеспечение,
% к вопросам поставки также относится и вопрос условий использования,
% или \emph{лицензирования продукта}. 

% Данный вопрос не относится напрямую к теме
% разработки драйверов и программного обеспечения в целом, однако
% является очень важным с точки зрения защиты авторских прав разработчика,
% поэтому в описательной части объектного файла .ko модуля ядра содержатся поля
% для указания автора, описания, а также лицензии, по которой распространяется данный модуль.

% В рамках рассмотрения лицензирования программного обеспечения следует отметить следующее:
% \begin{itemize}
%       \item в основном закрытые решения поставляются на основе \emph{авторского права (copyright)};
%       \item напротив, открытые решения поставляются на основе \emph{авторского лева (copyleft)}.
% \end{itemize}

% Отличительной особенностью лицензий авторского лева является то, 
% что в отличие от лицензий авторского права они в общем случае не накладывают
% ограничений на свободное распространение, вопроизведение и редактирование продукта,
% тем самым предоставляя сообществую возможность развивать продукт на некоммерческой основе,
% хотя в общем же случае и не запрещают коммерциализацию каких-либо изменений на основе данного продукта.

% Так как реализованный в рамках курсового проекта 
% драйвер разработан в учебных целях и не претендует на коммерческое использование,
% он поставляется в открытом виде под лицензией GPL, использованной по аналогии
% с исходным кодом ядра Linux.

% \subsection{Установка программного обеспечения}

% В предыдущем подразделе обсуждались вопросы, связанные с тем,
% в каком виде и на каких условиях может поставляться программное обеспечение, 
% однако был специально упущен момент, связанный со \emph{способом поставки}
% программного обеспечения. Данный факт связан с тем, 
% что помимо вопросов о источниках поставки,
% способ поставки также включает в себя вопрос \emph{способа установки продукта в систему}.

% Существуют следующие способы поставки продукта в систему:
% \begin{itemize}
%       \item с сайта разработчика;
%       \item с репозитория разработчика;
%       \item с использованием пакетного менеджера.
% \end{itemize}

% Первый способ является наиболее распространенным в экосистеме операционной
% системы Windows. Как правило, в данном случае предоставляется полноценный
% бинарный образ установщика, поставляющий в систему продукт, 
% а также необходимые для него зависимости.

% В Unix-подобных системах наиболее распространенным способом поставки программного
% обеспечения является последний вариант,
% так как наличие продукта в репозиториях пакетного менеджера полностью автоматизирует
% поставку продукта и зависимостей, а также предоставляет широкие возможности
% по согласованию версий и зависимостей в рамках конкретного дистрибутива системы.

% Данный проект выполнен в учебных целях и не ставит
% цели распространять модуль для последующего использования,
% поэтому в качестве способа поставки используется 
% открытый удаленный репозиторий.

% \subsection{Руководство по использованию драйвера}

Для тестирования разработанного в рамках проекта драйвера клавиатуры необходимо:
\begin{enumerate}
      \item Склонировать git-репозиторий с исходным кодом драйвера,
            расположенным по адресу: \url{https://github.com/Qurcaivel/usbkeyboard.git}.
      \item Собрать исходный код проекта согласно инструкции, приведенной
            в файле описания проекта readme.md.
      \item Протестировать полученный модуль с помощью одного из скриптов,
            расположенных в репозитории.
\end{enumerate}
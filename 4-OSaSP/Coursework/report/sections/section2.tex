\section{Системное проектирование} \label{system-design}

С точки зрения архитектуры ядра Linux, модуль ядра, драйвер, является специальным объектным файлом, 
динамически загружаемым в ядро для расширение его функциональности. 

При написании драйверов используется подход размещения кода в пределах одной единицы трансляции. 
Данный факт объясняется тем,
что, как правило, модули ядра разрабатываются под одну конкретную часть системы, а потому предоставляемый разработчику
программный интерфейс позволяет написать понятный, имеющий типизированную структуру код.
Кроме того, подобный подход во многом упрощает сборку модуля,
выполняемую с помощью системы Kbuild \cite{kbuild}.

Несмотря на вышесказанное, функции, из которых состоит модуль ядра,
всегда можно разделить на различные функциональные группы, выполняющие
условно изолированные друг от друга задачи.

Драйвер клавиатуры состоит из следующих групп функций:
\begin{itemize}
    \item группа вспомогательных функций;
    \item группа контрольных событий;
    \item группа управления нажатиями клавиш;
    \item группа управления состоянием светодиодов.
\end{itemize}

Разграничение на группы, представленное выше, не основывается на взаимодействии
с той или иной подсистемой, а осуществляется посредством разделения
функций драйвера как абстракции над реализацией: обратный вызов подсистемы
ввода окажется в одной группе с обратным вызовом подсистемы шины,
если оба обратных вызова отвечают за одну и ту же часть, например, 
за состояние светодиодов клавиатуры, и наоборот, два обратных вызова из одной подсистемы
окажутся в различных группах, если они обслуживают разные части.

% todo: сделать сслылку на приложение со структурной схемой

\subsection{Группа вспомогательных функций}

Данная группа функций отвечает за инициализацию и деинициализацию
полей структуры драйвера \texttt{usbkeyboard}.

\subsection{Группа контрольных событий}

Данная группа функций содержит в себе функции, отвечающие
за обработку различных событий, влияющих на состояние драйвера.

К этой группе относятся:
\begin{itemize}
    \item обратные вызовы подсистемы ввода: открытие и закрытие устройства ввода;
    \item обратные вызовы <<горячего>> подключения: подключение и отключение устройства.
\end{itemize}

\subsection{Группа управления нажатиями клавиш}

Данная группа функций выполняет обработку обратного вызова для запроса
состояния нажатия клавиш с последующей передачей сведений об изменении состояния
в подсистему ввода.

\subsection{Группа управления состоянием светодиодов}

Данная группа функций отвечает за установку состояния светодиодов на клавиатуре,
зависящего от подсистемы ввода.

Подход к изменению состояния светодиодов на клавиатуре как отдельном
физическом устройстве через обработку событий подсистемы ввода имеет
одну очень интересную особенность: поскольку события об изменении состояния
светодиодов поступают не от устройства ввода, <<связанного>> с конкретным физическим
устройством в контексте драйвера, а вообще от любого устройства ввода,
зарегистрированного в подсистеме, то и нажатие Lock-клавиши
влияет на состояние всех устройств, что используют данный подход,
даже если это это событие было вызвано программно, 
например, со стороны пользовательского пространства \cite{rubini}.
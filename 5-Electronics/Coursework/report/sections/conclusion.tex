\csection{Заключение}

В данном разделе будут подведены итоги по проектированию осциллографа с интерфейсом к персональному компьютеру.

В ходе выполнения курсового проекта были получены теоретические сведения, необходимые для построения схем основных преобразований аналогового сигнала, таких как деление, умножение и смещение.
Примененные при разработке схемы электротехнические решения являются типовыми для современных осциллографов, что также способствует пониманию их внутреннего устройства.

Разработанный USB-осциллогаф имеет два канала с диапазоном входного сигнала
\begin{itemize}
    \item от -25 до 25 В при использовании щупов 1:1;
    \item от -250 до 250 В при использовании щупов 1:10.
\end{itemize}
и позволяет исследовать часто встречаемые в любительской среде аналоговые сигналы, например, аудиосигналы, ультразвук, переменный ток в лабораторных цепях, показания датчиков.

Полученная схема устройства имеет простую структуру, поэтому без проблем может быть усовершенствована, например, путем добавления логического анализатора для исследования дискретных сигналов, или наращиванием входного каскада для увеличения допустимого диапазона исследуемого аналогового сигнала.
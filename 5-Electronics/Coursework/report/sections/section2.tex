\section{Сравнение особенностей конструкции с аналогами}

В качестве сравнения возьмем несколько коммерческих и любительских осциллографов (обозначены звездочкой):
\begin{itemize}
    \item Hantek DSO-6022BL;
    \item Актаком ADS-3025;
    \item OWON VDS1022I;
    \item DPScope (*);
    \item DPScope II (*).
\end{itemize}

Ниже представлены таблицы с основными сравнительными характеристиками перечисленных цифровых осциллографов.

\begin{table}[H]
    \caption{Сравнение характеристик коммерческих осциллографов}
    \label{tab:oscilloscope-comparison-1}
    
    \begin{tabularx}{\textwidth}{|l|X|X|X|}
        \hline
        Характеристика & Hantek DSO-6022BL & Актаком ADS-3025 & OWON VDS1022I \\
        \hline
        Ширина полосы пропускания & 20 МГц & 25 МГц & 25 МГц \\
        \hline
        Частота дискретизации & 48 МВыб/с & 100 МВыб/с & 100 МВыб/с \\
        \hline
        Количество каналов & 2 & 1 & 2 \\
        \hline
        Разрешение & 8 бит & 8 бит & 8 бит \\
        \hline
        Глубина памяти & 2 кБ & 6 кБ & 5 кБ \\
        \hline
        Подключение & USB 2.0 & USB 2.0 & USB 2.0 \\
        \hline
    \end{tabularx}
\end{table}

\begin{table}[H]
    \caption{Сравнение характеристик любительских осциллографов}
    \label{tab:oscilloscope-comparison-2}
    
    \begin{tabularx}{\textwidth}{|l|X|X|X|}
        \hline
        Характеристика & DPScope II & DPScope & Проект \\
        \hline
        Ширина полосы пропускания & 2.5 МГц & 1 МГц & 250 КГц \\
        \hline
        Частота дискретизации & 50 МВыб/с & 20 МВыб/с & 2 Мвыб/с \\
        \hline
        Количество каналов & 2 & 2 & 2 \\
        \hline
        Разрешение & 10 бит & 10 бит & 10 бит \\
        \hline
        Глубина памяти & 16 кБ & 16 кБ & 8 кБ \\
        \hline
        Подключение & USB 2.0 & USB 2.0 & USB 2.0 \\
        \hline
    \end{tabularx}
\end{table}

Исходя из данных, содержащихся в таблицах \ref{tab:oscilloscope-comparison-1}, \ref{tab:oscilloscope-comparison-2}, можно сделать следующие выводы:
\begin{enumerate}

    \item Осциллографы <<Актаком ADS-3025>> и <<OWON VDS1022I>> имеют наивысшую частоту дискретизации, значение которой влияет на детализацию получаемой осциллограммы, однако стоит иметь ввиду, что зачастую осциллографы используются для одновременного отслеживания нескольких сигналов, чего не может делать одноканальный <<Актаком>>, при этом за счет мультиплексирования выборки частота дискретизации уменьшается в два раза, что делает <<OWON>> сравнивым с любительским DPScope II.
    
    \item Любительские осциллографы заметно уступают коммерческим в частоте дискретизации, что в первую очередь объясняется используемыми в них из соображений экономии низкочастотными микроконтроллерами.
    
    \item Пропускная способность коммерческих осциллографов на порядок превосходит значения любительских осциллографов, поскольку в них используются дорогостоящие пассивные компоненты.
    
    \item Разрешение коммерческих осциллографов меньше чем у любительских, что может быть несущественной уступкой в пользу двукратного экономии памяти; любительские осциллографы не могут себе такого позволить ввиду более низкого параметра дискретизации.
\end{enumerate}

Помимо этого следует упомянуть, что в осциллографах <<Hantek DSO-6022BL>> и <<DPScope>> присутствует логический анализатор, что позволяет эффективнее анализировать цифровые сигналы.
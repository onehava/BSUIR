\section{Разработка структурной схемы}

Для построения структурной схемы необходимо выделить основные функции устройства. Затем нужно определить компоненты, выполняющие данные функции, и установив функциональную связь между этими компонентами, соединить их между собой.

Конечный результат разработки структурной схемы осциллографа можно увидеть в Приложении \ref{app:structure}.

\subsection{Основные функции устройства}

Осциллограф с интерфейсом к персональному компьютеру должен выполнять следующие функции:
\begin{enumerate}
    \item Предварительное преобразование входного сигнала (два канала).
    \item Обработка входного сигнала.
    \item Прием команд и передача данных на персональный компьютер.
\end{enumerate}

\subsection{Определение компонентов устройства}

В соответствие перечисленным функциям устройства можно сопоставить следующие компоненты:
\begin{enumerate}
    \item Модуль преобразования канала - масштабирует и смещает диапазон входного сигнала (два канала).
    \item Микроконтроллер - производит обработку входного сигнала, а также управление всей схемой.
    \item Интерфейс подключения - обеспечивает прием и передачу данных между микроконтроллером и персональным компьютером.
    \item Питание - фактически относится к интерфейсу подключения USB, но логически отделен от него ввиду выполнения иной функции.
\end{enumerate}

\subsection{Взаимодействие компонентов устройства}

Микроконтроллер имеет двунаправленную связь с модулями преобразования канала (два канала), а также интерфейсом подключения. 

Модуль питания имеет однонаправленную связь в сторону всех остальных компонентов устройства.
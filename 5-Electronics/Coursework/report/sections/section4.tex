\section{Разработка функциональной схемы}

Для построения функциональной схемы необходимо установить процесс, иллюстрируемый схемой. Затем нужно определить функциональные части изделия, участвующих в этом процессе, и установив связь между этими частями, соединить их между собой.

Конечный результат разработки функциональной схемы осциллографа можно увидеть в Приложении \ref{app:functional}.

\subsection{Определение иллюстрируемого процесса}

В случае осциллографа с интерфейсом к персональному компьютеру, функциональная схема должна иллюстрировать последовательное преобразование входного аналогового сигнала с его последующим преобразованием в цифровой вид и передачей на основное устройство.

\subsection{Определение функциональных частей схемы}

На основании установленного процесса можно выделить следующие функциональные части устройства:
\begin{enumerate}
    \item Аттенюатор 1:10 - уменьшает амплитуду смещенного входного сигнала в десять раз.
    \item Операционный усилитель x1 - буферизирует входной сигнал для передачи на АЦП микроконтроллера (большие сигналы).
    \item Операционный усилитель x10 - усиливает входной сигнал в десять раз для передачи на АЦП микроконтроллера (маленькие сигналы).
    \item Микроконтроллер - выполняет преобразование входного аналогового сигнала в дискретные значения.
\end{enumerate}

\subsection{Взаимодействие функциональных частей схемы}

Входной сигнал канала и сдвиговый сигнал приходят на вход аттенюатора, в результате чего на выходе получается отцентрированный сигнал большего диапазона.
С выхода аттенюатора сигнал идет на буферизирующий операционный усилитель x1, после чего направляется на операционный усилитель x10, уменьшающий диапазон сигнала.
С операционных усилителей сигнал подается на микроконтроллер, где дискретизируется, после чего направляется в интерфейс подключения.
\section{Разработка принципиальной схемы}

Конечный результат разработки функциональной схемы осциллографа можно увидеть в Приложении \ref{app:principled}.

\subsection{Входной каскад}

Аналого-цифровой преобразователь контроллера PIC18F13K50 имеет входной диапазон 0-5 В.
Сигналы, меньшие этого диапазона, будут измеряться с пониженным разрешением (поскольку они не охватывают весь диапазон возможных значений разрядов АЦП), а более крупные сигналы будут отсекаться до максимального значения \cite{pic18f13k50}.

Сначала во входном каскаде входящий сигнал ослабляется в 10 раз делителем напряжения, образованным резисторами R7 и R8. Помимо увеличения диапазона входного напряжения в десять раз, последовательное подключение этих резисторов дает сопротивление 1 МОм, достаточное для подключения ко входу пассивных щупов 1:10 и 1:100.

Поскольку последующие схемы не могут взаимодействовать с отрицательным напряжением (для простоты в схеме отсутствует отрицательное питание), для смещения отрицательной половины вверх используется низкоомный делитель на резисторах R1, R2, R3. Конденсатор C1 выступает в роли фильтра и гарантирует, что от быстрого изменения входного сигнала сигнал смещения не <<поплывет>>.

Емкостный делитель C3, C4 вместе с парой резисторов R1, R3 образует так называемый \emph{частотно-скомпенсированный делитель}. 
Причина его использования состоит в следующем: между входом (входным сигналом со щупа осциллографа) и АЦП микроконтроллера находятся элементы с некоторой паразитной емкостью (провода, диоды, операционные усилители). Если использовать только R1 и R3, в результате в схеме образуется RC-фильтр, что серьезно уменьшит пропускную способность осциллографа.
Однако, если выполняется равенство $(C_4 + C_{par})\cdot R_8 = C_3 \cdot R_7$, коэффициент деления от частоты сигнала зависеть не будет \cite{fcd}.

Диоды D1, D2 действуют как входная защита, ограничивая любые сигналы, поступающие на операционный усилитель и превышающие значения напряжения 0-5 В.

Перечисленные выше процессы описаны с использованием элементов первого канала осциллографа.

\subsection{Блок усиления}

Для усиления сигналов используется MCP6024, который содержит 4 операционных усилителя в одном корпусе. Главной особенностью данного элемента является то, что он является так называемым <<Rail-to-rail>> усилителем, т.е. он работает нормально, даже если входной сигнал или выходной сигнал доходит до его шин питания, что позволяет не беспокоиться об обработке предельных значений \cite{mpc6024}.

Каждый канал использует два операционных усилителя из <<пачки>>: один в качестве буфера между входным каскадом (импеданс которого превышает максимально допустимый импеданс источника для аналого-цифрового преобразователя микроконтроллера), второй для усиления сигнала в десять раз (можно использовать для маленьких сигналов).

\subsection{Триггер}

В триггере используется компаратор RC0, встроенный в микроконтроллер, что исключает необходимость использования внешних схем.
Триггер сравнивает масштабированный входной сигнал с пороговым уровнем, управляемым пользователем.
Для того, чтобы сгенерировать пороговый сигнал, можно использовать цифро-аналоговый преобразователь, однако для исключения ненужных задержек можно использовать имеющийся в микроконтроллере генератор широтно-импульсной модуляции RC5. Поскольку генерируемый сигнал представляет собой не постоянный уровень, а быструю прямоугольную волну, он подается на фильтр нижних частот C8, R17, временная константа которого подобрана таким образом, чтобы подаваемый на компаратор сигнал представлял собой среднее напряжение ШИМ.

\subsection{Блок подключения}

Микроконтроллер PIC18F14K50 имеет полноценный USB-интерфейс, поэтому с аппаратной точки зрения реализация подключения для передачи данных тривиальна.

Присутствует развязка источника питания — объемный конденсатор C13 в сочетании с обмоткой L1 и конденсатором С12 фильтрует питание: 
C13 действует как буфер для предотвращения скачков энергопотребления схемы, 
а C12 и L1 блокируют высокочастотный шум, поступающий от ПК к осциллографу, 
или помехи, исходящие от осциллографа.
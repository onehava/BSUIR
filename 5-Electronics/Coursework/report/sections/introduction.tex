\csection{Введение}

В современной электротехнике осциллографы являются неотъемлемой частью инструментария любого специалиста. В отличие амперметров, вольтметров и частотомеров, позволяющих узнать численные значения характеристик электрического сигнала, осциллографы предоставляют возможность его визуализации и анализа, что является критически важным для разработки, тестирования и диагностики электронных устройств.

Осциллографы позволяют инженерам наблюдать форму, амплитуду, частоту и временные характеристики электрических сигналов. Это необходимо для определения неисправностей, оценки качества сигналов, измерения параметров и последующей коррекции схемы.

В настоящее время как среди энтузиастов-электротехников, так и среди инженеров-специалистов, пользуются популярностью осциллографы с интерфейсом подключения к персональному компьютеру, среди которых заметно выделяются \emph{USB-осциллографы}. 
В отличие от полноценных осциллографов, данные устройства, как правило, имеют более низкие характеристики и возможности, однако являются более популярным решением в повседневном использовании, и вот почему:
\begin{enumerate}
    \item Портативность и компактность: USB-осциллографы довольно компактны и легко переносимы, что делает их идеальным вариантом для инженеров и электронщиков, работающих в разных местах.
    \item Удобство подключения: интерфейс USB обеспечивает простое подключение к большей части современной вычислительной техники без установки дополнительных карт расширения и сложных процедур настройки оборудования.
    \item Питание: интерфейс USB не только обеспечивает передачу данных, но и предоставляет питание устройству, что позволяет сократить число подключений осциллографа до одного.
    \item Программное обеспечение: многие USB-осциллографы поставляются с программным обеспечением, которое предоставляет широкие возможности анализа и представления данных.
    \item Доступность и цена: USB-осциллографы часто более доступны по цене по сравнению с традиционными стационарными осциллографами, что делает их привлекательным выбором для энтузиастов, студентов и начинающих специалистов в области электротехники.
\end{enumerate}

Данный курсовой проект направлен на разработку схемы собственного USB-осциллографа с использованием современных технологий и компонентов. 
Целью учебного проекта является разработка устройства, способного обеспечивать достойные для своей компонентной базы характеристики измерения входных сигналов.

В рамках работы будут рассмотрены основные технические решения в области строения цифровых осциллографов, названы причины их использования. Ожидается, что результаты данной работы могут оказаться полезны не только для желающих ознакомиться со строением цифровых осциллографов, но и для людей, увлекающихся аналоговой схемотехникой, так как многие компоненты, используемые в данном проекте, являются типовыми для данной предметной области.

Курсовая работа состоит из нескольких разделов, включающих вводную теоретическую часть, сравнение конструкционных особенностей с аналогами, проектирование структурных, функциональных и принципиальных схем устройства. По завершению работы будут сделаны выводы, которые обобщат приобретенные и актуализированные в ходе работы теоретические сведения и практические навыки.
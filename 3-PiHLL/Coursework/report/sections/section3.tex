\section{Структура входных и выходных данных}

\subsection{Входные данные}

Входные данные представлены текстовым файлом с исходным текстом программы на языке Бейсик.

Синтаксис данной реализации языка можно определить в виде расширенной формы Бэкуса-Наура
следующего вида:

\begin{Verbatim}[fontsize=\small]
    CHARS = ? все допустимые символы ?
    EOL   = ? символ окончания строки ?
    EOF   = ? символ окончания файла ?

    DIGIT = '0' | '1' | '2' | '3' | '4' | 
            '5' | '6' | '7' | '8' | '9'
    ALPHA = 'A' | 'B' | 'C' | 'D' | 'E' | 'F' | 'G' | 'H' | 
            'I' | 'J' | 'K' | 'L' | 'O' | 'M' | 'N' | 'P' | 
            'Q' | 'R' | 'S' | 'T' | 'U' | 'V' | 'W' | 'X' | 
            'Y' | 'Z'

    IDENTIFIER = (ALPHA) {ALPHA | DIGIT}
    INTEGER    = (DIGIT) {DIGIT}
    FLOATING   = (INTEGER) ['.'  INTEGER]
    STRING     = '"' {CHARS} '"' 

    PROGRAM  = COMPOUND
    COMPOUND = {LINE}
    LINE     = STATEMENT EOL
    
    STATEMENT = IF | FOR | WHILE | ASSIGN | INPUT | PRINT

    IF     = 'IF' BEXPR 'THEN' (LIF | MIF)
    LIF    = STATEMENT
    MIF    = EOL BEXPR ['ELSE' COMPOUND] 'ENDIF'

    FOR    = 'FOR' IDENT '=' NEXPR 'TO' NEXPR COMPOUND 'NEXT'
    WHILE  = 'WHILE' BEXPR COMPOUND 'WEND'

    ASSIGN = IDENT '=' NEXPR
    
    INPUT [STRING], IDENT
    PRINT (EXPR) {';' EXPR}
    
    SEXPR = ? строчные выражения ?
    NEXPR = ? числовные выражения ?
    BEXPR = ? логические выражения ?
\end{Verbatim}

Выражения в языке строятся согласно семантическим правилам операций,
перечисленных ниже:
\begin{itemize}
    \item Бинарные арифметические операции над числами.

    \item Бинарные операции логического сложения и умножения.

    \item Бинарные операции сравнения чисел.
    
    \item Унарные операции 'плюс' и 'минус' для чисел.

    \item Унарная операция логического отрицания.
    
    \item Числовые, логические, строковые терминалы.
    
    \item Скобочная вложенность.
\end{itemize}

\subsection{Выходные данные}

Выходными данными интерпретатора можно считать
результат выполнения операций \emph{Input} и \emph{Print},
возможные сообщения об ошибках,
а также внутреннее состояние памяти, отвечающей за хранение
переменных программы.
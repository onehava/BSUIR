%\section{Алгоритмы разработанных методов}
\section{Функциональное проектирование}

\subsection{Схемы алгоритмов}

Для построения блок-схем алгоритмов возьмем приватные функции-члены класса
\emph{Parser}.

\subsubsection{Алгоритм функции Parser::parse\_if()}.
Эта функция является функцией-членом класса \emph{Parser}
и выполняет синтаксический анализ и построение объекта класса
If : public Statement, представляющего в памяти оператор ветвления.

В пояснительной записке блок-схема алгоритма этой функции
представлена в \mbox{Приложении \ref{bss} (чертеж ГУИР.400201.001 ПД)}. 

\subsubsection{Алгоритм функции Parser::parse\_for()}.
Эта функция является функцией-членом класса \emph{Parser}
и выполняет синтаксический анализ и построение объекта класса
For : public Statement, представляющего в памяти оператор цикла с шагом.

В пояснительной записке блок-схема алгоритма этой функции
представлена в \mbox{Приложении \ref{bss} (чертеж ГУИР.400201.001 ПД)}.

\subsection{Алгоритмы по шагам}

Для описания алгоритмов по шагам возьмем приватные функции-члены класса
\emph{Lexer}.

\subsubsection{Алгоритм функции Lexer::read\_number() по шагам}.
Данная функция-член класса \emph{Lexer} выполняет лексический анализ
по итератору \emph{Lexer::m\_current} и возвращает тип числа,
полученного в результате обработки 
всей правильной последовательности символов.

Алгоритм по шагам имеет следующий вид:
\begin{enumerate}
    \item Пока текущее значение - цифра, сдвигать итератор вправо;
    \item Если текущее значение - точка, перейти к пункту \ref{pf};
    \item Вернуть значение Lexeme::Integer;
    \item Пока текущее значение - цифра, сдвигать итератор вправо \label{pf};
    \item Вернуть значение Lexeme::Floating;
    \item Конец.
\end{enumerate}

\subsubsection{Алгоритм функции Lexer::read\_unique() по шагам}.
Данная функция-член класса \emph{Lexer} выполняет лексический анализ
по итератору \emph{Lexer::m\_current} и возвращает тип слова,
полученного в результате обработки 
всей правильной последовательности символов.

Алгоритм по шагам имеет следующий вид:
\begin{enumerate}
    \item Пока текущее значение итератора - буква или цифра, 
          сдвигать итератор вправо;

    \item Вызвать специальную функцию,
          которая определит тип ключевого слова,
          заключенного между указателями 
          предыдущего и текущего состояния итератора.

    \item Вернуть результат вызова специальной функции,
          являющийся либо типом ключевого слова, 
          либо Lexeme::Ident;

    \item Конец.
\end{enumerate}
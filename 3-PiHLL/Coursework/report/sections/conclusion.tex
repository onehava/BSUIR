\csection{Заключение}

В данном разделе будут подведены итоги 
по разработке программы интерпретатора языка Бейсик.

В ходе работы по написанию курсового проекта 
были изучены некоторые теоретические сведения
из информатики, связанные с теорией формальных языков,
а также основные подходы к описанию и классификации
языков программирования, их синтаксиса и грамматики.

Был разработан собственный диалект языка Бейсик, включающий
в себя основные инструкции ранних версий языка,
на котором можно было бы написать несложные программы,
работающие с вещественными числами,
а также операциями ввода/вывода с клавиатуры.

Была реализована программа интерпретатора, которая может
выполнить любую программу, 
написанную на разработанном диалекте,
а также указывающая на пользователю
на ошибки в случае нарушения основных правил языка.

В процессе разработки использовались основные возможности
языка программирования C++, а именно:
\begin{enumerate}
    \item объектно-ориентированное программирование;
    \item система шаблонных типов;
    \item элементы стандартной библиотеки STL.
\end{enumerate}

Разработка проекта осуществлялась с использованием системы
контроля версий Git для разрешения возможных
конфликтов при разработке модулей проекта,
а также система сборки проектов CMake для удобного управления
правилами сборки и управления зависимостями.

Выполнение задачи курсового проекта позволяет приобрести
новые навыки в объектно-ориентированном проектировании
приложений, а также увеличить опыт в написании
программ с использованием методологии объектно-ориентированного
программирования.

Кроме того, в ходе написания кода на языке C++ возникает
необходимость в параллельном изучении продвинутых
возможностей языка, что помогает актуализировать
знания о методологии разработки программного обеспечения
и использовать современные подходы в решении
соответствующих задач.
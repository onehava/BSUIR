\section{Постановка задачи}

Программа интерпретатора должна принимать на вход
текстовый файл, содержащий исходный код программы на 
языке Бейсик, производить соответствующий анализ,
сообщать пользователю о различных ошибках,
а также исполнять программу в соответствии с правилами
языка.

Необходимо разработать иерархию классов, связанных между
собой механизмами агрегации и наследования.

При написании модулей программы следует использовать
доступные инструменты стандартной библиотеки STL:
алгоритмы, адаптеры, контейнеры, умные указатели,
систему исключений.

Основной функционал интерпретатора должен обеспечивать:
\begin{itemize}
    \item Лексический анализ исходной программы
          с разбиением текста на правильные лексемы.

    \item Синтаксический анализ промежуточного представления
          программы с построением синтаксической структуры.
          
    \item Семантический анализ синтаксической структуры для
          проверки правильности поведения исходной программы.
          
    \item Интерпретация конечного представления в соответствии
          с правилами языка.
\end{itemize}

Сообщения об ошибках должны быть наглядными, а поведение
интерпретатора ожидаемым и предсказуемым.
\csection{Введение}

Язык программирования C++ --- компилируемый, статически
типизированный язык программирования общего назначения.

Поддерживает такие парадигмы программирования как
процедурное программирование
объектно-ориентированное программирование,
метапрограммирование.

Язык обеспечивает модульность программы, а также его раздельную
компиляцию. На высоком уровне реализует систему исключений,
абстракцию данных, а также полиморфные типы
(виртуальные функции и виртуальное наследование).

Стандартная библиотека языка STL (Standart Template Library)
основывается на одной из механик языка, системе шаблонных типов,
и содержит в себе обобщенные типы и функции, 
реализованные согласно общим подходам
объектно-ориентированного проектирования,
основными из которых являются
контейнеры, итераторы и алгоритмы.

С каждым новым стандартом язык приобретает все больше высокоуровневых
абстракций для взаимодействия на различных системных уровнях.

В стандарте С++98 были добавлены шаблонные типы,
библиотека STL с реализацией основных контейнеров,
строковых типов, а также потоков ввода/вывода.

В стандарте С++11 была добавлена механика перемещения,
лямбда-функции, система многопоточности std::thread,
модель памяти, умные указатели, хеш-таблицы,
а также поддержка регулярных выражений.

В стандарте C++17 было добавлено множество языковых конструкций,
способных оптимизировать процесс компиляции,
а также упростить написание высоких абстракций с низкой или 
нулевой стоимостью (zero-cost abstractions).
В стандартную библиотеку была включена универсальная библиотека 
взаимодействия с файловой системой std::filesystem,
а также безопасные типы std::optional и std::variant.

В стандарте C++20 для улучшения разработки
обобщенных типов на базе шаблонов были добавлены концепты (concept),
библиотека алгоритмов абстракций с нулевой стоимостью std::ranges,
поддержка на уровне языка примитивов асинхронного программирования,
атомарных типов для многопоточного программирования,
а также система модулей для разделения зависимостей на независимые
встраиваемые компоненты.

Учитывая все вышеперечисленные факты, несложно заметить
что язык C++ постоянно притерпевает все новые и новые изменения.
Новые стандарты расширяют набор технологий языка, 
тем самым обеспечивая его актуальность, 
а также гибкость как инструмента разработки общего назначения. 

Таким образом, с одной стороны, язык C++ сохраняет некоторую
обратную совместимость кода с кодом, написанным на языке C,
а с другой стороны - предоставляет множество высокоуровневых
инструментов для разработки эффективных программ.

Область применения языка C++ включает в себя создание
операционных систем, написание драйверов устройств,
приложений для встраиваемых систем, высоконагруженных приложений,
а также прикладных программ и сервисов общего назначения.

На основании изложенного материала можно сделать вывод,
что использование языка C++ для реализации программы
по теме данного курсового проекта является весьма актуальным.
Высокоуровневость языка позволяет рассматривать входные 
данные на высоком уровне абстракции, при этом
поддержка на уровне языка некоторой низкоуровневости 
позволяет писать оптимальный код 
для оптимального решения специфических задач.

Темой данного курсового проекта является разработка программы 
интерпретатора языка программирования BASIC. 
Актуальность данного проекта обусловлена активным развитием 
в отрасли информационных технологий множества новых технологий 
в разработке и проектировании приложений. 
Для их поддержания разрабатываются новые языки программирования, 
которые были бы способны предоставлять разработчикам способ 
быстрой и эффективной разработки и обслуживания 
программного кода под конкретную поставленную задачу. 

Наглядным примером такого языка является язык Python, 
который на сегодняшний день является одним из самых востребованных 
в таких областях как сетевое программирование и машинное обучение.
Факт того, что основная редакция языка представляет собой
исполнительную машину интерпретирующего типа, создает необходимость
в понимании принципов работы интерпретаторов, а также их основных
отличий от других видов схожих по назначению программ.
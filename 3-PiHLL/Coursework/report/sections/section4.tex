%\section{Описание классов}
\section{Структурное проектирование}

\label{class-description}

Перед тем как начать описывать каждый класс по отдельности, стоит прокомментировать
общую архитектуру программы. 

При проектировании данной программы учитывалась требовательность к производительности
итоговой исполнительной машины, а также необходимость в поддержании архитектурной 
целостности проекта при постепенном расширении его функционала.

С этой целью было принято решение не писать <<жадный>> интерпретатор, сразу приступающий
к исполнению <<сырого>> текста исходной программы, а применить некоторые существующие
паттерны объектно-ориентированного проектирования, 
которые используются при создании полноценных языков программирования.

Для этого логика программы разделена на независимые модули, выполняющие преобразование
исходной программы на соответствующем им уровне абстракции данных.

Основными такими модулями являются модули \emph{лексического}, \emph{синтаксического},
и \emph{семантического} анализа, последовательно преобразующие
получаемые данные в данные более высокого уровня абстракции.

После завершения преобразования данные передаются в \emph{исполнительный} модуль,
реализованный с использованием паттерна проектирования <<Интерпретатор>>.

UML-диаграмма архитектуры проекта интерпретатора
представлена в \mbox{Приложении \ref{uml} (ГУИР.400201.001 PР)}

\subsection{Класс Core - ядро программы}

Ниже представлен листинг объявления класса Core
в заголовочном файле <<Core.h>>:

\begin{verbatim}
class Core {
public:
    static void run(const std::string& directory);

private:
    static void run_program(std::string&& program);
};
\end{verbatim}

Метод void run(const std::string\& directory)
отвечает за запуск программы, 
записанной в файле, расположенном по пути directory.

Метод void run\_program(std::string\&\& program)
отвечает за исполнение исходного кода
программы, представленного в виде строки program.

\subsection{Класс Monitor - монитор ошибок}

Ниже представлен листинг объявления класса Monitor
в заголовочном файле <<Monitor.h>>:

\begin{verbatim}
class Monitor {
public:
    using process_t = std::function<void(void)>;

    Monitor(Source& source, process_t process);
    void run();
    
private:
    process_t m_process;
    Source& m_source;

    void indicate(std::size_t row, 
                  std::size_t pos,
                  const std::string& message);

    void underline(std::size_t row, 
                   std::size_t pos1, std::size_t pos2, 
                   const std::string& message);
    };    
\end{verbatim}

Конструктор Monitor(Source\&, process\_t)
инициализирует ссылку на адаптер исходного кода программы,
а также функцию process, 
исполнение которой будет <<мониторить>> данный класс.

Методы void indicate(...) и void underline(...)
являются вспомогательными методами вывода декорированных
сообщений об ошибках: indicate указывает на позицию в строке,
а underline - на последовательность символов.

\subsection{Класс Error - ошибка языка}

Заголовочный файл Error содержит объявление класса 
Error как унаследованной от std::exception <<пустышки>>,
а также производных от него классов.

Ниже представлен листинг объвлений унаследованных
от Error классов ошибок в заголовочном файле <<Error.h>>:

\begin{verbatim}
class SyntaxError : public Error {
public:
    SyntaxError(std::size_t row, std::size_t pos);

    std::size_t get_row() const;
    std::size_t get_pos() const;

private:
    std::size_t m_row;
    std::size_t m_pos;
};
\end{verbatim}

Класс SyntaxError образует подмножество классов
синтаксических ошибок и содержит в себе поля для хранения
позиции в файле, где эта ошибка произошла.

\begin{verbatim}
class UnknownSymbolError : public SyntaxError {
public:
    UnknownSymbolError(std::size_t row, 
                       std::size_t pos, char symbol);

    char get_symbol() const;

private:
    char m_symbol;
};
\end{verbatim}

Класс UnknownSymbolError содержит в себе поля 
для хранения информации об ошибке <<неизвестный символ>>.

\begin{verbatim}
class UnterminatedStringError : public SyntaxError {
public:
    UnterminatedStringError(std::size_t row, std::size_t pos);
};
\end{verbatim}

Класс UnterminatedStringError содержит в себе поля
для хранения информации об ошибке <<незавершенная строка>>.

\begin{verbatim}
class TypeError : public Error {
    ...
};
\end{verbatim}

Класс TypeError является <<пустышкой>> и выделяет 
среди ошибок класса Error подмножество ошибок,
связанных с типом выражения.

Данное подмножество ошибок используется при семантическом
анализе выражений для обозначения неверных операций.

\begin{verbatim}
class WrongExprTypeError : public TypeError {
public:
    WrongExprTypeError(Lexeme first, 
                       Lexeme last, 
                       Value::Type unexpected, 
                       Value::Type expected);

    Lexeme get_first() const;
    Lexeme get_last() const;
    Value::Type get_expected() const;
    Value::Type get_unexpected() const;

private:
    Lexeme m_first, m_last;
    Value::Type m_expected, m_unexpected;
};
\end{verbatim}

Класс WrongExprTypeError содержит в себе поля 
для хранения информации об ошибке <<неверный тип выражения>>.

\begin{verbatim}
class UnknownUnaryExprError : public TypeError {
public:
    UnknownUnaryExprError(Lexeme op, Value::Type type);
    Lexeme get_operation() const;
    Value::Type get_type() const;

private:
    Lexeme m_op;
    Value::Type m_type;
};
\end{verbatim}

Класс UnknownUnaryExprError содержит в себе поля
для хранения информации об ошибке <<неизвестная унарная операция>>
и используется на этапе семантического анализа при обнаружении
неверных унарных операций.

\begin{verbatim}
class UnknownBinaryExprError : public TypeError {
public:
    UnknownBinaryExprError(Lexeme op, 
                           Value::Type left, 
                           Value::Type right);

    Lexeme get_operation() const;
    Value::Type get_left_type() const;
    Value::Type get_right_type() const;

private:
    Lexeme m_op;
    Value::Type m_left, m_right;
};            
\end{verbatim}

Класс UnknownBinaryExprError содержит в себе поля
для хранения информации об ошибке <<неизвестная бинарная операция>>
и используется на этапе семантического анализа при обнаружении
неверных бинарных операций.

\subsection{Класс Source - контейнер программы}

Ниже представлен листинг объявления класса Source
в заголовочном файле <<Source.h>>:

\begin{verbatim}
class Source {
public:
    class iterator;

    Source(std::string text);

    Iterator begin();
    Iterator end();

    Line get_by_index(std::size_t index) const;

private:
    std::string m_text;
    std::vector<Line> m_content;
};
\end{verbatim}

Данный класс выполняет функцию адаптера для исходного кода
программы, представляя данные таким образом,
чтобы соответствовать основных характеристикам языка,
а также содержит в себе вспомогательные адаптеры
строк для предоставления операции вывода
строки из текста по её номеру.

Конструктор Source(std::string) принимает исходный код программы,
приводит все не-строковые записи к верхнему регистру для обеспечения
регистронезависимости языка при определении его ключевых слов,
а также создает набор адаптеров строк исходного кода.

Метод begin() предоставляет доступ к содержимому класса
и возвращает итератор на начало исходного кода программы.

Метод end() предоставляет доступ к содержимому класса 
и возвращает итератор на конец исходного кода программы.

Метод get\_by\_index(std::size\_t) принимает номер 
строки в исходном коде и случае её наличия возвращает
адаптер строки, в случае отсутствия - вызывает
исключение.

Ниже представлено объявление
классов Line и Source::iterator в том же
заголовочном файле:

\begin{verbatim}
class Line {
public:
    friend std::ostream& operator<<(std::ostream& ostream, 
                                    const Line& line);

private:
    friend class Source;
    Line(const char* pointer);
    const char* const m_pointer;
};

class Source::Iterator {
public:
    Iterator();
    const char& operator*() const;
    Iterator& operator++();
    Iterator operator++(int);

    bool operator!=(const Iterator& other) const;
    bool operator==(const Iterator& other) const;

    const char* get_ptr() const;
    std::size_t get_row() const;
    std::size_t get_pos() const;

private:
    friend class Source;
    Iterator(const char* pointer);

    const char* m_cur;
    std::size_t m_row;
    std::size_t m_pos;
};
\end{verbatim}

Класс Line является вспомогательным и существует для того
чтобы выводить строку из исходного кода программы без 
восстановления её длины и потерь памяти на хранение.

Класс Iterator позволяет перемещаться по исходному коду
программы, хранящемся в адаптере Source. Особенностью его
реализации является operator++(), содержащий в своей логике
подсчет номера строки, а также номера текущего символа.

\subsection{Класс Lexeme - лексема языка}

Ниже представлен листинг объявления класса Lexeme
в заголовочном файле <<Lexeme.h>>:

\begin{verbatim}
class Lexeme {
public:
    enum Type : uint8_t {...};

    Lexeme();
    Lexeme(Type type, std::string_view view, 
           std::size_t row, std::size_t pos);

    Type get_type() const;
    bool has_type(Type type) const;

    std::string_view get_view() const;
    std::size_t get_row() const;
    std::size_t get_pos() const;

private:
    Type m_type;
    std::string_view m_view;
    std::size_t m_row, m_pos;
};
\end{verbatim}

Класс Lexeme хранит в себе тип лексемы, установленный
лексическим анализатором, std::string\_view для
быстрого доступа к представлению лексемы 
в исходном коде, а также номер строки и позицию символа,
на которых находится начало объекта лексемы.

\subsection{Класс Lexer - лексический анализатор языка}

Ниже представлен листинг объявления класса Lexer
в заголовочном файле <<Lexer.h>>:

\begin{verbatim}
using Lexemas = std::vector<Lexeme>;

class Lexer {
public:
    Lexer() = default;
    Lexemas tokenize(Source::iterator source);

private:
    Source::iterator m_curr;
    Source::iterator m_prev;

    bool match(char required);

    Lexeme next();
    Lexeme make(Lexeme::Type type);

    Lexeme::Type read_lexeme();
    Lexeme::Type read_number();
    Lexeme::Type read_string();
    Lexeme::Type read_unique();
};
\end{verbatim}

Метод tokenize(Source::iterator) является основным методом
класса и отвечает за преобразование исходного кода в набор
отдельных лексем.

Метод next() пропускает все пробельные символы и комментарии
начиная с текущей позиции, после чего при помощи
методов read\_x() осуществляет 
лексический анализ текущей последовательности символов
и возвращает полученную лексему.

Методы read\_x() отвечают за обработку соответствующих 
лексических примитивов языка, выполняют поглощение их 
символов, а также установления их типов.

Метод match(char) является вспомогательным методом
и используется в методах чтения для сокращения
записи по выявлению типа текущего символа и его поглощения
в случае совпадения.

Метод make(Lexeme::Type) является вспомогательным
и используется для создания
полученной в результате текущей итерации
синтаксического анализа лексемы.

\subsection{Класс Expression - выражение языка}

Класс Expression является базовым типом для определения
структур представления выражений языка в памяти.

Ниже представлен листинг объявления класса Expression
в заголовочном файле <<Expression.h>>:

\begin{verbatim}
class Expression {
public:
    enum class Type : uint8_t {
        Unary, Binary, Numeric, String, Ident,
    };

    Expression(Type type);
    Type type() const;

private:
    Type m_type;
};
\end{verbatim}

Класс Expression предоставляет возможность установить
тип производного класса по полю базового класса
Expression::m\_type.

Файл содержит объявления следующих производных классов:
\begin{itemize}
    \item Unary - хранит Lexeme::Type унарной операции
          и владеет указателем на другое выражение.

    \item Binary - хранит Lexeme::Type бинарной операции
          и владеет указателями на два других выражения.

    \item Numeric - хранит значение типа double для
          сокращения представления от Lexeme к бинарному
          виду.
          
    \item String и Ident - владеют std::string,
          единственная разница между ними - в типе
          выражения, которое они собой представляют.
\end{itemize}

Данный подход в наследовании с сопоставлением
переменной типа используется как более удобная
в данном контексте альтернатива виртуальному наследованию
с восстановлением типа производного класса через
паттерн проектирования <<Visitor>>.

Также в этом же файле содержится объявление адаптера 
доступа к набору выражений Expressions:

\begin{verbatim}
using Expression_ptr = std::unique_ptr<Expression>;

class Expressions {
public:
    class iterator;
    void push_back(Expression_ptr expression);
    
    iterator begin() const;
    iterator end() const;

private:
    using container_t = std::vector<Expression_ptr>;
    container_t m_container;
};

class Expressions::iterator {
public:
    iterator() = default;
    const Expression& operator*() const;
    iterator& operator++();

    bool operator==(const iterator& other) const;
    bool operator!=(const iterator& other) const;

private:
    friend class Expressions;
    iterator(container_t::pointer pointer);

    container_t ::pointer m_pointer;
};
\end{verbatim}

Вся суть данного адаптера состоит в том,
что он обеспечивает сокрытие деталей реализации 
хранения объектов Expression в памяти
и предоставляет доступ к ним
не через указатели, а через константные ссылки.

Также адаптер содержит метод push\_back(...)
для возможности создания набора выражений
без операций копирования.

\subsection{Класс Statement - оператор языка}

Класс Statement является базовым типом для определения
структур представления операторов языка в памяти.

Ниже представлен листинг объявления класса Statement
в заголовочном файле <<Statement.h>>:

\begin{verbatim}
class Statement {
public:
    enum class Type : uint8_t {
        Assign, If, For, While, Input, Print,
    };

    Statement(Type type);
    Type type() const;

private:
    Type m_type;
};
\end{verbatim}

Как несложно заметить, класс Statement образует такую же
иерархию наследования и представления в памяти как и 
класс Expression.

Для него объявлен адаптер Statements для сокрытия
памяти хранения объединеных операторов,
реализованный точно также как и адаптер Expressions.

Файл содержит объявления следующих производных классов:
\begin{itemize}
    \item Assign - владеет std::string для хранения 
          идентификатора и указателем на выражение, 
          которое этому идентификатору присваивается.

    \item If - владеет указателем на выражение,
          представляющее собой логическое условие
          оператора ветвления,
          а также два адаптера Statements:
          один для ветки if, другой - для ветки else.

    \item For - владеет std::string для хранения 
          идентификатора переменной, по которой 
          цикл итерирует,
          двумя указателями на выражения,
          которые представляют собой начальное
          и конечное значения для итерации,
          а также хранит адаптер Statements,
          который представляет собой тело цикла.

    \item While - владеет указателем на выражение,
          представляющее собой логическое условие
          цикла с условием,
          а также хранит адаптер Statements,
          который представляет собой тело цикла.

    \item Input - владеет двумя std::string:
          в одной хранится приглашение на ввод,
          а в другой - идентификатор, который необходимо
          ввести.

    \item Print - хранит адаптер Expressions,
          который содержит в себе выражения, которые
          нужно последовательно вывести на экран.
\end{itemize}

\subsection{Класс Value - значение переменной}

Ниже представлен листинг объявления класса Value 
в заголовочном файле <<Value.h>>:

\begin{verbatim}
class Value {
public:
    enum Type : uint8_t {
        Numeric, Boolean, Literal,
    };

    Value(bool v);
    Value(double v);
    Value(std::string v);

    Type get_type() const;

    bool to_boolean() const;
    double to_numeric() const;
    const std::string& to_literal() const;

private:
    Type m_type;
    std::variant<bool, double, std::string> m_content;
};
\end{verbatim}

Данный класс представляет собой представление
значения выражения во времени исполнения.

Для класса определены конструкторы из языковых типов,
а также методы взятия значений по типу.

\subsection{Класс Constexpr - семантическая обертка выражения}

Ниже представлен листинг объявления класса Constexpr
в заголовочном файле <<Constexpr.h>>:

\begin{verbatim}
class Constexpr {
public:
    static Constexpr build(Lexeme);
    static Constexpr build(Lexeme, Constexpr);
    static Constexpr build(Lexeme, Constexpr, Constexpr);

    Expression_ptr get_expression();
    Value::Type get_type() const;

private:
    Constexpr(Expression_ptr, Value::Type);
    Constexpr(Lexeme::Type, Constexpr, Value::Type);
    Constexpr(Lexeme::Type, Constexpr, Constexpr, Value::Type);

    Expression_ptr m_expr;
    Value::Type m_type;
};
\end{verbatim}

Данный класс является <<умным>> типом данных, и используется
на этапе парсинга для совмещения синтаксического
и семантического анализа выражений.

Рассмотрим конструкторы класса:
\begin{itemize}
    \item Constexpr(Expression\_ptr, Value::Type) - создает
          объект, инициализируя его поля по аргументам
          конструктора.
          
    \item Constexpr(Lexeme::Type, Constexpr, Value::Type) -
          создает объект, инициализируя поле выражения 
          как унарную операцию от Lexeme::Type и выражения 
          внутри Constexpr-аргумента.

    \item Constexpr(Lexeme::Type, Constexpr, Constexpr, Value::Type) -
          создает объект, инициализируя поле выражения
          как бинарную операцию от Lexeme::Type и выражений
          внутри Constexpr-аргументов.
\end{itemize}

Данные конструкторы являются вспомогательными и могут
использоваться лишь внутри класса, так как не проверяют
правильность операции по соответствию типов. 

Описанную выше задачу выполняют статические методы build():
\begin{itemize}
    \item build(Lexeme) - выполняет построение выражения 
          со значением согласно правилам языка.
        
    \item build(Lexeme, Constexpr) - выполняет построение
          унарного выражения согласно правилам языка.

    \item build(Lexeme, Constexpr, Constexpr) - выполняет
          построение бинарного выражения согласно правилам языка.
\end{itemize}

В случае несоответствия типов операции, методы build(...)
вызывают соответствующие исключения, 
описанные в данном разделе выше.

\subsection{Класс Parser - синтаксический анализатор языка}

Ниже представлен листинг объявления класса Parser
в заголовочном файле <<Parse.h>>:

\begin{verbatim}
using Lexemas = std::vector<Lexeme>;

class Parser {
public:
    Parser() = default;
    Statements parse(Lexemas::iterator source);

private:
    Lexemas::iterator m_current;

    template<typename... Types>
    bool match(Types... types);

    inline const Lexeme& previous() const;
    inline const Lexeme& current() const;
    inline void require(Lexeme::Type type);
    void parse_compound(Statements& statements);

    Statement_ptr parse_statement();
    std::unique_ptr<If> parse_if();
    std::unique_ptr<For> parse_for();
    std::unique_ptr<While> parse_while();
    std::unique_ptr<Assign> parse_assign();
    std::unique_ptr<Input> parse_input();
    std::unique_ptr<Print> parse_print();

    Expression_ptr parse_expression(Value::Type type);

    Constexpr expression();

    Constexpr disjunction();
    Constexpr conjunction();

    Constexpr comparison();

    Constexpr term();
    Constexpr factor();

    Constexpr unary();
    Constexpr primary();
};
\end{verbatim}

Данный класс отвечает за синтаксический анализ программы
и построения ее абстрактного синтаксического дерева,
в данном проекте представленного в виде списка
операторов языка, которые могут содержать
вложенности, тем самым образуя древовидное ветвление связей.

Метод parse(Lexemas::iterator) отвечает за разбор программы
и в случае успеха возвращает список операторов Statements.

Класс содержит в себе три основные группы функций:
\begin{enumerate}
    \item Вспомогательные - используются для чтения лексем.
    \item Разбора операторов - разбирают лексемы на операторы.
    \item Разбора выражений - используют метод рекурсивного
          спуска для формирования правильных выражений.
\end{enumerate}

Вспомогательные функции:
\begin{itemize}
    \item match(Types... types) - в случае совпадения
          типа текущей лексемы с одним из аргументов
          поглощает ее и возвращает истину, иначе - ложь.

    \item previous() - возвращает ссылку на предыдущую лексему.
    
    \item current() - возвращает ссылку на текущую лексему.
    
    \item require(Lexeme::Type type) - требует соответствия
          текущей лексемы типу аргумента.
\end{itemize}

Функции разбора операторов:
\begin{itemize}
    \item parse\_compound(Statements\&) - 
          выполняет последовательный разбор 
          операторов и их запись по ссылке аргумента.

    \item parse\_statement() - выполняет разбор оператора.
          Определяет тип текущего оператора и возвращает
          результат посредством делегирования полномочий
          по разбору на другие parse-методы класса.

    \item parse\_if() - выполняет разбор оператора ветвления.
    \item parse\_for() - выполняет разбор цикла с шагом.
    \item parse\_while() - выполняет разбор цикла с условием.
    \item parse\_assign() - выполняет разбор оператора присваивания.
    \item parse\_input() - выполняет разбор оператора ввода.
    \item parse\_print() - выполняет разбор оператора вывода.
\end{itemize}

Как уже было упомянуто выше, функции разбора выражений 
используют метод рекурсивного спуска.

Суть данного метода заключается в следующем:
\begin{itemize}
    \item Каждая функция отвечает за соответствующие
          операции со своими особыми свойствами
          (арность, ассоцитивность, приоритет).

    \item Порядок вызова функций определяется в первую
          очередь приоритетом соответствующих ей операций.

    \item Разбор выражения начинается с вызова
          первой определенной по порядку функции.

    \item Функция производит соответствующие проверки
          и в соответствии со своей логикой возвращает
          либо конечное выражение, либо результат вызова
          следующей по порядку функции.
          
    \item Таким образом, каждая функция <<пробрасывает>>
          через себя вызов других функций,
          в результате чего каждая вызванная функция
          владеет данными о соответствующей им операции.

    \item Разбор завершается сверткой <<проброса>>,
          когда вызываемые функции перестают вызывать 
          функции более низкого порядка.
\end{itemize}

\subsection{Класс Registry - адаптер для хранения переменных}

Ниже представлен листинг объявления класса Registry
в заголовочном файле <<Registry.h>>:

\begin{verbatim}
class Registry {
public:
    Value& operator[](const std::string& ident);

private:
    std::unordered_map<std::string, Value> m_vars;
};
\end{verbatim}

Функция-член класса operator[~](const std::string\&)
предоставляет доступ к значению переменной по её названию,
возвращая изменяемую ссылку на объект.

\subsection{Класс Evaluator - вычислитель выражений}

Ниже представлен листинг объявления класса Evaluator
в заголовочном файле <<Evaluator.h>>:

\begin{verbatim}
class Evaluator {
public:
    Evaluator(Registry& registry);
    Value evaluate(const Expression& expr);

private:
    Registry& m_registry;

    inline Value evaluate(const Unary& unary);
    inline Value evaluate(const Binary& binary);
    inline Value evaluate(const Numeric& numeric);
    inline Value evaluate(const String& string);
    inline Value evaluate(const Ident& ident);
};      
\end{verbatim}

Класс Evaluator по своей структуре напоминает
реализацию паттерна проектирования <<Посетитель>>,
поскольку содержит в себе приватные функции,
отвечающие за посещение выражений соответствующего типа.

Конструктор Evaluator(Registry\&) связывает объект класса
Evaluator с объектом класса Registry для вычисления
выражений с переменными величинами.

Ниже представлено описание функций вычисления:
\begin{itemize}
    \item evaluate(const Expression\& expr) -
          определяет тип текущего выражения и возвращает
          результат его вычисления посредством делегирования
          вычисления другим функциям вычисления.

    \item evaluate(const Unary\&) -
          вычисляет унарное выражение посредством
          применения операции по отношению к результату
          вызова функции evaluate(...) для вложенного выражения.

    \item evaluate(const Binary\&) -
          вычисляет бинарное выражение посредством
          применения операции по отножению к результатом
          вызовов функции evaluate(...) для вложенных выражений.

    \item evaluate(const Numeric\&) -
          возвращает численное значение,
          хранящееся в выражении типа Numeric.
          
    \item evaluate(const String\&) -
          возвращает строковое значение,
          хранящееся в выражении типа String.

    \item evaluate(const Ident\&) - 
          возвращает выражение, хранящееся
          в реестре переменных под названием идентификатора.  
\end{itemize}

\subsection{Класс Interpreter - интерпретатор программы}

Ниже представлен листинг объявления класса Interpreter
в заголовочном файле <<Interpreter.h>>:

\begin{verbatim}
class Interpreter {
public:
    Interpreter();
    void execute(const Statements& statements);

private:
    Registry registry;
    Evaluator evaluator;

    void execute(const Statement& stmt);
    inline void execute(const If& stmt);
    inline void execute(const For& stmt);
    inline void execute(const While& stmt);
    inline void execute(const Assign& stmt);
    inline void execute(const Input& stmt);
    inline void execute(const Print& stmt);
};
\end{verbatim}

Класс Interpreter по своей структуре напоминает
реализацию паттерна <<Посетитель>>,
и имеет реализацию, аналогичную классу Evaluator.

Ниже представлены функции исполнения:
\begin{itemize}
    \item execute(const Statements\&) - производит
          последовательное исполнение операторов.

    \item execute(const Statement\&) -
          определяет тип текущего оператора и производит 
          его исполнение посредством делегирования
          другим функциям вычисления.

    \item execute(const If\&) -
          исполняет оператор \emph{If}.

    \item execute(const For\&) - 
          исполняет оператор цикла \emph{For}.

    \item execute(const While\&) -
          исполняет оператор цикла \emph{While}.

    \item execute(const Assign\&) - 
          исполняет оператор \emph{Let}.

    \item execute(const Input\&) -
          исполняет оператор \emph{Input}.

    \item execute(const Print\&) - 
          исполняет оператор \emph{Print}.
\end{itemize}
\section{Теоретические сведения}

Для понимания того, как реализовать свой язык программирования,
необходимо понимать, что такое \emph{язык программирования}.

Данный раздел содержит краткие сведения о языках программирования,
а также об их сходствах и различиях.

%\subsection{Понятие языка программирования}
\subsection{Основные понятия из теории языков программирования}

\emph{Язык программирования} -- это формальный язык,
который может использоваться для записи компьютерных программ.

В информатике под \emph{формальным языком} понимают некоторое
множество конечных слов над конечным алфавитом.

Язык программирования определяет набор 
\emph{лексических}, \emph{синтаксических} и \emph{семантических} 
правил, которые определяют внешний вид программы, а также действия, 
которые она будет исполнять.

\emph{Лексика} определяет множество слов языка, т.е. то,
как из алфавита языка образуются определенные слова.

\emph{Синтаксис} определяет набор правил, по которым
описываются комбинации слов языка,
считающихся правильно структурированными программами.

\emph{Семантика} определяет набор правил, которые 
придают смысл синтаксически правильным программам.

Итак, язык программирования -- это совокупность правил,
определяющих структуру программы, слов, из которых она состоит,
а также то, что на основании этой программы будет выполнять
некоторая исполнительная машина.

Помимо основных правил, на которых строится язык,
существует множество специальных терминов, дающих ему
некоторую условную характеристику, 
например, \emph{динамическая типизация}, 
или \emph{явное преобразование типов}, однако все они
не входят в тему данной работы. Более важным для нас являются 
не столько термины из области \emph{стандартизации} 
языков программирования, сколько
термины и определения из области их \emph{реализации}.

%\subsection{Понятие транслятора}

Наиболее распространенной классификацией реализаций языков
программирования по типу исполнения является
разделение языков на \emph{интерпретируемые} и \emph{компилируемые}.

Интерпретаторы и компиляторы, ответственные за соответствующие
им процессы \emph{интерпретации} и \emph{компиляции},
относятся к одному более общему классу программ -- \emph{трансляторам}.

\emph{Транслятор} -- это техническое средство, выполняющее трансляцию,
т.е. преобразование исходной программы на одном из языков программирования,
в программу, написанную на другом языке.

Суть операции трансляции заключается в том, что программа на неизвестном
для исполнительной машины языке переводится в понятнее для нее представление,
после чего становится возможным исполнение на этой машине исходной программы.

\emph{Компиляция} -- это техника выполнения, при которой 
исходный код программы на исходном языке транслируется,
обычно, в программу на языке более низкого уровня.

Когда мы говорим, что программа является \emph{компилятором},
мы подразумеваем что она переводит программу в какое-то
другое представление, но не исполняет ее. 
Пользователь получает результат \emph{компиляции}
и исполняет его самостоятельно, используя имеющиеся у него
для этого инструменты.

Когда же мы говорим, что программа является \emph{интерпретатором},
мы имеем ввиду то, что она получает на вход исходный код программы
и немедленно приступает к его исполнению, то есть 
запускает программу "из исходников".

Однако исполнение программы "из исходников" не означает,
что интерпретатор не преобразует, то есть не \emph{транслирует}
исходный код программы в иное представление.

В качестве примера можно рассмотреть основную реализацию языка
Python - интерпретатор CPython.
С точки зрения пользователя CPython действительно является
интерпретатором: он принимает на вход программу и сразу же
приступает к ее исполнению.
Однако с точки зрения технической реализации CPython содержит
в себе множество деталей, присущих компиляторам.

Он разбирает исходную программу и преобразует ее в древовидную
структуру (AST - Abstract Syntax Tree),
затем анализирует полученное представление и транслирует его 
в более простое промежуточное (IR - Intermediate Representation),
роль которого выполняет байткод,
после чего исполняет полученный байткод посредством JIT-компиляции.

В книге Роберта Нистрема <<Создание интерпретаторов>> вопрос
о разнице между интерпретаторами и компиляторами сравнивается
с вопросом о разнице между фруктами и овощами.
Несмотря на то, что выбор между фруктами и овощами выглядит
как взаимоисключающий, слово \emph{фрукт} является ботаническим
термином, а слово \emph{овощ} -- термином из кулинарии.
Таким образом всегда найдется овощ, который не является фруктом,
например, \emph{морковь}, но это вовсе не означает, что не найдется 
съедобного растения, являющегося и фруктом, и овощем, 
например \emph{томата}.

%\subsection{Краткие сведения об языке Basic}
\subsection{Краткие сведения об языке Бейсик}

Бейсик (BASIC, 
сокращение от англ. Beginner's All-purpose Symbolic Instruction Code 
-- универсальный код символических инструкций для начинающих) 
-- название семества языков высокого уровня.

Был разработан в 1964 году как инструмент,
при помощи которого студенты-непрограммисты
имели бы возможность создавать компьютерные программы для решения
собственных задач. 

Получил широкое распространение в качестве 
\emph{языка для домашнего компьютера}.

Внешний вид программ написанных на ранних версиях Бейсика
во многом определялся тем, что он предназначался для 
среды программирования со строковым редактором текста.
В таком редакторе пользователь не имел возможности отображать
весь текст на экране, а также перемещаться по нему в любых
направлениях с помощью перефирийных устройств.
В строковых редакторов для редактирования строки текста
пользователь должен был дать команду изменения строки с заданным
номером, а затем ввести новый текст указанной строки.
Для вставки новой строки нужно было дать команду вставки,
указав правильный номер.
Вводимые таким образом строки можно было отобразить на экране
согласно их нумерации при помощи специальной команды \emph{List}.

%\subsection{Описание синтаксиса языка Бейсик} % ???

Как уже было сказано ранее, на синтакис Бейсика главным образом 
повлияла среда программирования, на которой он использовался.

Ранние версии языка содержали очень малое число ключевых слов:
их количество могло не достигать и 20.

Программа языка была представлена набором строк,
пронумерованных целыми числами, называемыми метками строк.

Далее строка содержала инструкцию языка, которая завершалась 
символом перехода на новую строку. Существовали реализации,
которые позволяли помещать несколько инструкций на одной строке,
разделяя их символом двоеточия <<:>>.

Ниже представлен список основных инструкций:

\begin{itemize}
    \item <<Input ["Приглашение"], Переменная>> -- вывод на экран
          опционального приглашения на ввод и чтение
          значения с клавиатуры в переменную.
          
    \item <<Print 
          (\emph{Текст} | \emph{Переменная}) 
          ; ... ; 
          (\emph{Текст} | \emph{Переменная})>> --
          вывод перечисляемых значений на экран.

    \item <<Let \emph{Переменная} = \emph{Значение}>> -- операция 
          присваивания значения переменной, 
          ключевое слово Let в большинстве реализаций
          предполагалось как опциональное.

    \item <<Goto \emph{Метка}>> -- безусловный переход на строковую метку.
    
    \item <<If \emph{Условие} Then ...> -- условный оператор.
    
    \item <<While \emph{Условие}>> -- заголовок цикла с условием.
    
    \item <<Endw> -- завершение тела цикла.
    
    \item <<For \emph{Переменная} = \emph{Начало} 
          To \emph{Конец} [Step \emph{Шаг}]>> -- заголовок
          цикла с шагом, параметр Step являлся опциональным и 
          в случае его отсутствия шаг цикла равнялся 1.
\end{itemize}

В реализациях языка с интерактивным режимом поддерживались
команды, необходимые для взаимодействия исполняемой
среды с пользовательским терминалом, а также команды взаимодействия
с файловой системой и перефирийными устройствами.

Также ранние реализации языка содержали специфические ограничения:
длина идентификатора переменной не превышала одного символа,
а типизация обеспечивалась специальными суффиксами
('\$' для строковых типов, '\%' для целочисленных и т.д.).
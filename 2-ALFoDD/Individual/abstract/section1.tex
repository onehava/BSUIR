\section{Машинное обучение}

\subsection{Понятие машинного обучения}

Машинное обучение -- подраздел искусственного интеллекта, включающий в себя
методы, характерной чертой которых является не прямое решение поставленной
задачи, а обучение за счет применения решений множества сходных задач.

Различают два типа обучения:
    
\begin{itemize}
      \item Индуктивное обучение (обучение по прецедентам) основывается
            на выявлении эмпирических закономерностей в данных.

      \item Дедуктивное обучение предполагает формализацию знаний экспертов
            и их представление в виде базы знаний.
\end{itemize}

Дедуктивное обучение принято относить к области экспертных систем,
поэтому понятия обучения по прецедентам и машинного обучения считают синонимами.

Машинное обучение находится на стыке таких предметных областей как математическая статистика,
численные методы, методы оптимизации, математический анализ, но также имеет и собственные области, 
связанные с проблемами вычислительной эффективности и переобучения.

\subsection{Постановка задачи обучения по прецедентам}

Имеется множество \emph{объектов} (ситуаций), множество возможных \emph{ответов} (реакций)
и существует некоторая неизвестная зависимость между ответами и объектами.
Известно конечное множество прецендентов -- пар (объект, ответ), называемое \emph{обучающей выборкой}.
Требуется по этим \emph{частным} данным выявить \emph{общие} закономерности, присущие не только
определенной выборке, но и всем прецедентам, в том числе тем, что еще не наблюдались.

\subsection{Преимущества машинного обучения}

% Подход, основанный на решении задачи за счет обучения на основе некоторого набора прецедентов,
% создает ряд существенных преимуществ методов машинного обучения по сравнению со
% многими альтернативными методами, такими как ручной анализ, жестко запрограммированные правила
% и простые статистические модели.

Подход, лежащий в основе методов машинного обучения и заключающийся в решении задач
за счет обучения по некоторому набору прецедентов, создает ряд существенных преимуществ
перед многими общепринятыми альтернативными методами, такими как ручной анализ данных,
жестко запрограммированные правила и простые статистические модели.

Среди них можно выделить данные преимущества:

\begin{itemize}
      \item Точность -- по мере обучения системы происходит накопление данных, что
      повышает точность принимаемых решений.

      \item Автоматизация -- модель, основанная на методах машинного обучения
      может автоматически обнаруживать новые шаблоны.

      \item Скорость -- машинное обучение позволяет получать точные ответы быстрее,
      что дает возможность системам реагировать в реальном времени.

      \item Масштабируемость -- модель, основанная на методах машинного обучения
      легко приспосабливается к увеличению объема данных.
\end{itemize}

\subsection{Классические задачи машинного обучения}

Задача кластеризации -- задача, которая предполагает упорядочивание объектов 
в сравнительно однородные группы \cite{mandel}. Решением задачи является алгоритм,
группирующий исходное множество объектов на подмножества (кластеры) таким образом, чтобы
объекты из одного кластера имели большее сходство, чем объекты из разных кластеров.

Задача классификация -- задача, в которой определено некоторое начальное множество
\emph{объектов}, разделенных по определенным признакам на \emph{классы}, называемое
\emph{выборкой}. Решение задачи предполагает создание алгоритма, способного 
классифицировать объект, то есть, устанавливать соотвествие между ним и определенным классом.

Задача регрессии -- задача, предполагающая исследования зависимости одной зависимой переменной
от конечного набора некоторых независимых переменных \cite{foerster}. Целью регрессионного анализа является
предсказание значения зависимой переменной по набору независимых, а также определение
вклада отдельных независимых переменных в изменение зависимой.

Задача понижения размерности данных -- задача по сведению большого количества
признаков к меньшему, более удобного для дальнейшего использования и представления. 
Примером задачи понижения размерности является сжатие данных.

\subsection{Способы машинного обучения}

Раздел машинного обучения, с одной стороны, основывается на методах математической 
статистики, с другой -- на моделировании механизмов обработки информации 
в биологических нейронных сетях.

Ниже приведены некоторые основные способы машинного обучения, используемые для
решения классических задач:

\begin{itemize}
      \item Обучение с учителем -- испытуемая система принудительно обучается
      с помощью примеров, образующих \emph{обучающую выборку}.

      \item Обучение без учителя -- испытуемая система спонтанно обучается 
      решать поставленную задачу.

      \item Обучение с подкреплением -- испытуемая система (\emph{агент}), обучается,
      взаимодействуя с окружением, называемым \emph{средой}.
\end{itemize}

Выбор способа машинного обучения во многом зависит от поставленной задачи.
Задача кластеризации, как правило, решается при помощи статистических методов, 
например, метода k-средних, основанном на пересчете центров кластеров. 
В задачах классификации и регрессии используют способ обучения с учителем, 
а в задаче понижения размерности данных используется способ обучения без учителя. 
Наиболее наглядным примером обучения с подкреплением являются генетические алгоритмы, 
основанные на биологических механизмах приспособления к окружающей среде. 
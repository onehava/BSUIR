\csection{Заключение}

В результате выполнения работы по учебной практике были представлены результаты
теоретического исследования по теме <<Методы машинного обучения
в обработке звуковых и речевых сигналов>>. Были представлены общие
сведения о методах машинного обучения и их применения в обработке
звука и речи. 

В ходе написания реферата были изучены основы машинного обучения.
Были выявлены основные преимущества использования нейросетового подхода,
приведены примеры нейросетевых архитектур, используемых
в обработке звуковых и речевых сигналов.
В качестве примера была рассмотрена задача шумоподавления
и методы её решения, приведены примеры использования
активного шумоподавления в системах голосовой связи и сообщения.
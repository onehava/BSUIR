\csection{Введение}

% Учебная практика является обязательным элементом подготовки специалиста с высшим 
% образованием и одной из форм текущей аттестации студента по учебной дисциплине. 
% Для студентов это первая учебная работа такого рода и объёма. 
% Реферат содержит результаты теоретического исследования по теме «Концепция Интернет вещей (IoT)», что способствует получению базовых навыков по поиску и систематизации информации, а также подготовке к дальнейшему изучению технологий, которые относятся к концепции IoT.


% В первом разделе дается базовая информация о концепции интернета вещей, 
% а также возможные области применения. 
% Описываются задачи, которые помогает решать данная область, возможные способы реализации.


% Второй раздел описывает основные характеристики технологии IoT, 
% методы использования устройств, которые можно отнести к заданной концепции, 
% а также части, без которых концепцию невозможно будет реализовать. 
% Третий раздел информирует о преимуществах технологии, а так же о недостатках. 


% Четвертый и пятый разделы описывают возможные реализации устройств экосистемы IoT,
% а также используемое программное обеспечение, 
% которое заложено в реализацию концепции IoT.


% Раздел шесть и семь содержат информацию об используемых протоколах передачи информации 
% между устройствами в сети, а так же более подробно описывают аспекты безопасности 
% системы интернета вещей.

Учебная ознакомительная практика является неотъемлемой частью образовательного 
процесса в высших учебных заведениях.

Целью ознакомительной практики является выработка профессиональных навыков
в поиске и обработке информации, а также её обработки и представлении
в соотвествии с указанными требованиями и стандартами.

Данный реферат содержит результаты теоретического исследования по теме
<<Методы машинного обучения в системах обработки звуковых и речевых сигналов>>,
которая является одним из перспективных направлений для решения 
существующих практических задач.

В первом разделе содержатся базовые понятия из области машинного обучения.
Дается представление о задаче обучения по прецендентам, перечисляются основные преимущества
методов машинного обучения перед вычислительными методами, а также перечисляются классические задачи машинного 
обучения и способы машинного обучения, используемые при решении данных задач.

Во втором разделе представлен материал по представлению звуковых сигналов для их обработки
в системах, использующих методы машинного обучения.

В третьем разделе дается представление о методах машинного обучения,
применяемых в обработке звуковых и речевых сигналов. 
Описывается архитектура нейронных сетей прямого распространения, а также классический метод их обучения.
Рассматривается архитектура сверточных нейронных сетей, принцип их действия,
основные преимущества их использования.

Четвертый раздел посвящен одной из основных задач обработки звуковых сигналов -- шумоподавлению.
В разделе содержится постановка задачи шумоподавления, а также примеры некоторых
нейросетевых архитектур, используемых для решения данной задачи. 
\csection{Введение}

Курсовая работа является неотъемлемой частью образовательного процесса в высших учебных заведениях.

Целью курсовой работы является контроль знаний студента, полученных в течение курса по соответствующей дисциплине,
актуализация знаний для перехода на следующую ступень обучения,
а также формирование соответствующих теоретических знаний и практических навыков, необходимых будущему специалисту.

Данная курсовая работа содержит теоретические и практические задачи по программе первого и второго семестра
курса дисциплины <<Арифметические и логические основы цифровых устройств>>.

Первый и второй разделы посвящены арифметическим и логическим основам в проектировнии цифровых устройств.
В рамкам первого задания рассматривается алгоритм вычитания чисел с плавающей запятой, а также выполнение практической
задачи согласно данному алгоритму. Во втором задании требуется минимизировать логическую функцию используя один из основных
методов минимизации.

Третий, четвертый и пятый разделы посвящены построению структурных схем цифровых устройств. 
В рамках третьего задания требуется синтезировать комбинационную схему демультиплексора.
В четвертом задании требуется представить теоретические сведения о цифровых автоматах.
В пятом задании необходимо синтезировать структурную схему цифрового автомата.
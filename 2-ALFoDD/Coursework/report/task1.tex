\section{Двоичная арифметика}

\subsection{Постановка задачи}

Требуется представить алгоритм вычитания двоичных чисел, преставленных в форме с плавающей запятой,
а также выполнить соответствующее арифметическое действие согласно алгоритму над двумя десятичными
числами $A = 10$, $B = 6$.

\subsection{Описание алгоритма вычитания чисел}

Алгоритм сложения (вычитания) чисел с произвольными знаками состоит в следующем~\cite{lua}:

\begin{enumerate}
    \item Произвести выравнивание порядков $p_A$ и $p_B$. Для этого из порядка числа $A$ вычитается порядок числа $B$. 
    Если $p = p_A - p_B > 0$, то $p_A > p_B$ и для выравнивания порядков необходимо сдвинуть вправо мантиссу $m_B$. 
    Если $p = p_A - p_B < 0$, то $p_B > p_A$, и для выравнивания порядков необходимо сдвинуть вправо мантиссу $m_A$. 
    Если $p = p_A - p_B = 0$, то $p_A= p_B$, и порядки слагаемых выравнивать не требуется.

    \item Выполнить сдвиг соответствующей мантиссы на один разряд, повторяя его до тех пор, пока $p \neq 0$.

    \item Выполнить сложение (вычитание) мантисс $m_A$ и $m_B$ по правилу сложения правильных дробей.
    
    \item Если при сложении (вычитании) мантисс произошло переполнение, то необходимо произвести нормализацию результата 
    путем сдвига мантиссы вместе со знаковым разрядом вправо на один разряд с увеличением порядка на единицу. 
    Если же произошла денормализация, то выполнить сдвиг мантиссы результата на соответствующее количество разрядов 
    влево с соответствующим уменьшением порядка суммы.

    \item Конец алгоритма.
\end{enumerate}

\subsection{Представление чисел в форме с плавающей запятой}

Так как в ЭВМ достаточно сложно выполнить операцию вычитания чисел в прямых кодах, вычитание рассматривается как сложение
положительного и отрицательного чисел, представленных в дополнительных кодах.

\vspace{1em}

$m_A = +0,1010 \hspace{1ex}, \hspace{1ex} p_A = +0100.$

$m_B = -0,1100 \hspace{1ex}, \hspace{1ex} p_B = +0011.$

\vspace{1em}

$\acode{m_A} = 0,1010 \hspace{1ex}, \hspace{1ex} \acode{p_A} = 0,0100.$

$\acode{m_B} = 1,0100 \hspace{1ex}, \hspace{1ex} \acode{p_B} = 0,0011.$

\subsection{Вычитание чисел}

Определим разность порядков чисел:

\vspace{1em}

$\acode{p} = \acode{p_A} + \acode{-p_B} = 0,0100 + 1,1101 = 0,0001.$

\vspace{1em}

Так как $p > 0$, сдвигу подвергается мантисса числа $B$:

\vspace{1em}

$\acode{m_B} = 1,0100$.

$\acode{m_B} = 1,\mathbf{1}010, \hspace{1ex}
\acode{p} + \acode{-1} = 0,0001 + 1,1111 = 0,0000 = 0.$
  
\vspace{1em}

Выполним сложение мантисс:

\vspace{1em}

$\acode{m_A} = 1,1010$.

$\acode{m_B} = 1,1010$.

$\acode{m_C} = \acode{m_A} + \acode{m_B} = 0,1010 + 1,1010 = 0,0100$. 

\vspace{1em}

Определеим порядок результата:

\vspace{1em}

$\acode{p_C} = max(\acode{p_A}, \acode{p_ B}) = \acode{p_A} = 0,0100$.

\vspace{1em}

Результат сложения денормализован, необходимо произвести нормализацию мантиссы:

\vspace{1em}

$\acode{m_C} = 0,1000, \hspace{1ex} \acode{p_C} = 0,0011.$

\vspace{1em}

Результат выполнения операции вычитания над данными числами, 
представленными в форме с плавающей запятой:

\vspace{1em}

$A - B = C = 0,1000_{2} \times 2^{3} = 4_{10}$.
\csection{Заключение}

В ходе проделанной курсовой работы были систематизированы знания по курсу дисциплины <<Арифметические и логические основы цифровых устройств>>.

В первом разделе был рассмотрен алгоритм сложения (вычитания) чисел в форме с плавающей запятой,
являющийся частью арифметических основ курса.
Во втором разделе для минимизации логической функции четырех переменных был использован метод
минимизирующих карт Вейча (Карно).

В четвертом разделе были изложены теоретические сведения из области абстрактных цифровых автоматов: способы описания 
их структуры и законов функционирования.

В третьем и пятом разделах курсовой работы были использованы теоретические знания курса дисциплины,
а также практические навыки в области синтеза структурных схем цифровых устройств: демультиплексора и автомата Мура.

На основании проделанной работы можно сделать вывод, что для качественного выполнения поставленной задачи 
необходимы не только практические навыки, но и наличие систематизированных знаний в данной области.
Курс АиЛОЦУ включает в себя множество фундаментальных понятий и сведений, без которых невозможно
досконально понимать принцип работы цифровых устройств, а также приобретать новые знания
в области электротехники.